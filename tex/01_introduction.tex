\chapter*{Введение}
\addcontentsline{toc}{chapter}{Введение}

С момента первоначального создания логическое программирование было признано одной из парадигм с наибольшим потенциалом автоматизированного использования параллелизма. Ковальски в своей книге определяет распараллеливание как сильную сторону логического программирования \cite{kowalski}. Это стало началом интенсивного развития данной области.

За последние два десятилетия были проведены новые исследования, в которых изучалась роль новых параллельных архитектур в ускорении и развитии новых подходов логического вывода на основе графического процессора \cite{logic_researchers2}. 

Целью проводимой работы является обзор методов и способов распараллеливания логического вывода.
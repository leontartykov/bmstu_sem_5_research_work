\addcontentsline{toc}{chapter}{Список литературы}
\renewcommand\bibname{Список литературы}
\bibliographystyle{utf8gost705u}  % стилевой файл для оформления по ГОСТу

\begin{thebibliography}{5}
	\makeatletter
	\def\@biblabel#1{#1. }
	\bibitem{kowalski}
	Kowalski R. Logic for Problem Solving. Elseiver North Holland //Inc. – 1979.
	
	\bibitem{logic_researchers2}
	Dovier A. Formisano A. Gupta G. Hermenegildo M. Pontelli E. Rocha R. Parallel Logic Programming: A Sequel //arXiv preprint arXiv:2111.11218. - 2021.
	
	\bibitem{logic_researchers1}
	Shapiro E. Y. Alternation and the computational complexity of logic programs //The Journal of Logic Programming. - 1984. – Т. 1. – №. 1. - С. 19-33.
	
	\bibitem{horn_disjunction}
	Адаменко А. Н., Кучуков А. М. Логическое программирование и Visual Prolog //СПб.: БХВ-Петербург. - 2003. - Т. 992.
	
	\bibitem{machine_logic_output}
	Тутынин В. А. Применение многоагентного подхода в проектировании архитектур машины логического вывода //Информационные технологии. - 2019. – С. 113-113.
	
	\bibitem{prolog}
	Bramer M. Logic programming with Prolog. – Secaucus : Springer, 2005. - Т. 9. – С. 10.5555.
	
	\bibitem{datalog}
	Abiteboul S., Vianu V. Datalog extensions for database queries and updates //Journal of Computer and System Sciences. – 1991. - Т. 43. – №. 1. – С. 62-124.
	
	\bibitem{yedalog}
	Chin B. et al. Yedalog: Exploring knowledge at scale //1st Summit on Advances in Programming Languages (SNAPL 2015). - Schloss Dagstuhl-Leibniz-Zentrum fuer Informatik, 2015.
	
	\bibitem{logic_researchers_3}
	Мельцов В. Ю., Симонов А. И. Разработка мультиагентной системы для моделирования параллельного логического вывода //Общество, наука, инновации (НПК-2013). - 2013. - С. 950-952.
	
	\bibitem{logic_researchers_4}
	Томчук М. Н., Страбыкин Д. А., Агалаков Е. В. Метод параллельного логического вывода следствий для исчисления высказываний //Программные продукты и системы. – 2012. – №. 2. – С. 142-144.
	
	\bibitem{parallel_logic}
	Conery J. S. Parallel execution of logic programs. - Springer Science \& Business Media, 2012. – Т. 25.
	
	\bibitem{warren}
	Gabriel J. R. et al. A tutorial on the warren abstract machine for computational logic. – 1985.
	
	\bibitem{mercury}
	Conway T. Towards parallel Mercury. - University of Melbourne, Department of Computer Science and Software Engineering, 2002.
	
	\bibitem{eam}
	Lopes R., Costa V. S., Silva F. A design and implementation of the Extended Andorra Model1 //Theory and Practice of Logic Programming. – 2012. – Т. 12. - №. 3. - С. 319-360.
	
	\bibitem{mercury_new}
	Bone P. Automatic parallelisation for Mercury : дис. – 2012.
	
	\bibitem{costa}
	Costa V. S., Dutra I., Rocha R. Threads and or-parallelism unified //Theory and Practice of Logic Programming. – 2010. – Т. 10. – №. 4-6. – С. 417-432.
	
	\bibitem{rosha}
	Santos J., Rocha R. On the implementation of an Or-parallel prolog system for clusters of multicores //Theory and Practice of Logic Programming. – 2016. – Т. 16. – №. 5-6. – С. 899-915.
	
	\bibitem{cuda}
	Козлов С. О., Медведев А. А. Использование технологии cuda при разработке приложений для параллельных вычислительных устройств //Вестник Курганского государственного университета. - 2015. – №. 4 (38). – С. 106-112.
	
	\bibitem{nvidia_titan_x}
	Bianco S. et al. Benchmark analysis of representative deep neural network architectures //IEEE Access. – 2018. – Т. 6. – С. 64270-64277.
	
	\bibitem{opencl}
	Munshi A. The opencl specification //2009 IEEE Hot Chips 21 Symposium (HCS). – IEEE, 2009. – С. 1-314.
	
	\bibitem{gpu_datalog}
	Alberto Martinez Angeles C. A. et al. A Datalog Engine for GPUs. -– 2014.
	
	\bibitem{yap_prolog}
	Costa V. S., Rocha R., Damas L. The YAP prolog system //Theory and Practice of Logic Programming. - 2012. – Т. 12. – №. 1-2. – С. 5-34.
	
	\bibitem{fast_gpu_over_cpu}
	Козина А. В. и др. Сравнение эффективностей многопоточных реализаций алгоритма шифрования RSA на GPU и CPU //Электронный журнал: наука, техника и образование. – 2018. – №. СВ1. – С. 124-132.
	
	
\end{thebibliography}
	